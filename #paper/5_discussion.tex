%%%%%%%%%%%%%%%%%%%%%%%%%%%%%%%%
\section{Discussion} \label{S:discussion}
%%%%%%%%%%%%%%%%%%%%%%%%%%%%%%%%
%
As stated in previous paragraphs, the main focus of the present research was to investigate the speech intelligibility levels of normal hearing (NH) versus hearing-impaired children with cochlear implants (HI/CI). For that purpose, we set up an experiment where ten utterances, from thirty two children, were transcribed and transformed into entropy measures, which served as our outcome variable.

Rather than use the common multilevel approach, the current research proposed a statistical method novel to the field of linguistics. First, we modeled the \textit{speech intelligibility} as a latent variable that can be inferred from the entropy replicates \cite{Everitt_1984, Hoyle_et_al_2014}. The previous not only allow us to `construct' an \textit{intelligibility} score, but also to control for all the variability assumed present in our data. We argue the former was important as it allowed us to make comparisons at appropriate levels, while the latter was important to not `manufacture' false confidence in our parameter estimates, ultimately leading us to incorrect inferences \cite{McElreath_2020}. Second, we used Directed Acyclic Graphs (DAG) \cite{Pearl_2009, Cinelli_et_al_2021} to describe our causal and non-causal hypothesis about the \textit{intelligibility} of speech, and determine which variables were relevant for its development. Finally, we implemented the method under the Bayesian framework, where we provided the steps and code for its implementation.
%
\begin{comment}
	we believe that pre-aggregating procedures could be more pernicious for a proper statistical inference \citep{McElreath_2020}
\end{comment}
	
The statistical results highlighted two specific points. First, and most important, considering the large variability registered at the children and replicates levels, we were not able to produce unequivocal null hypothesis rejections, at a $95\%$ confidence level. As a result, in some way, we were impeded to clearly delimit the magnitude and direction of some effects. Furthermore, simulation studies revealed the need for larger sample sizes in order to produce such decisive results.

Nevertheless, as a second point, we have shown that by using the posterior distribution of the estimates, we were able to provide `preliminary' evidence on some of our causal hypothesis. On the one hand, we found there was mild evidence towards a difference in \textit{intelligibility} between \textit{hearing status} groups, in favor of NH children. However, the evidence also revealed the difference was mostly observed between NH and HI/CI children with etiologies other than genetic. In that sense, we further hypothesized that since hearing loss is a brain issue, rather than ear issue \cite{Flexer_2011}, these children might be `re-wiring' their auditory neural pathways during an important maturing stage of the brain cortex (first twelve months) \cite{Flexer_2011}, and that is what causes the delay in language development. It is fair to say then, the researchers believe it would be interesting to further inspect this hypothesis, with the support of Magnetic Resonance Imaging (MRI).

On the other hand, our analysis clearly identified \textit{hearing age} as one of the main determinant in the evolution of \textit{speech intelligibility}. However, the story produced by our models was more nuanced, as they also indicated there might be a different evolution in \textit{intelligibility} between NH and HI/CI children. The last result gave evidence contrary to what was previously found \cite{Boonen_et_al_2021}. Furthermore, as in the case of \textit{hearing status}, we hypothesized this outcome have two possible complementary explanations, either: (i) the children develop their language differently, at different stages of their (\textit{hearing}) age \cite{Flexer_2011}, or (ii) given that the hearing signal provided by the apparatus is degraded \cite{Drennan_et_al_2008}, HI/CI children might take longer to achieve similar levels of \textit{intelligibility} than their NH counterparts. Therefore, the researchers believe it would be interesting to assess the latter hypothesis, and for that purpose, further collection of data would be required, i.e we would need to identify the hearing signal degradation for the cochlear implants.

Furthermore, our analysis provided `mild' evidence that HI/CI children with more severe hearing loss, as accounted by the \textit{pure tone average}, develop their language at slower rate than their NH analogues. We further noticed, the previous result couples rather well with the second hypothesis proposed to explain the effects of \textit{hearing age}, i.e. the more degraded the hearing signal, either because of the apparatus or the severity of the hearing-impairment, the slower the \textit{intelligibility} evolves.

As perspectives for future research and further critiques, two finals points can be highlighted. First, the procedure for calculating the entropy replicates weighted equally any differences in the transcriptions. This implied the entropy values were equally increased by fairly small deviances, e.g. kikker `\textit{frog}' versus kikkers `\textit{frogs}', than with considerable ones, e.g. mismatches as in jongen `\textit{boy}' versus hond `\textit{dog}'. In that sense, the researchers also share the believe of \citet{Boonen_et_al_2021}, that a further refinement of the entropy calculation can be made, in which the linguistic distance of the transcriptions can be considered.

Second and last, although we used a DAG to state our causal and non-causal hypothesis, it is clear, the benefits of the tool shine when it is also used to plan the data collection. As stated in the document, the tool not only helps to make our hypothesis more transparent, but also allow us to derive statistical procedures from such assumptions. Scenarios like the lack of power to test specific effects, and the need for larger samples could have been easily foresaw during this part of the experimental planning \cite{McElreath_2020, Fogarty_et_al_2022}.
%
\begin{comment}	
	and remains to determined if these observations were a result of the actual variability in the data or an artifact of the measurement procedure?
\end{comment}
%
%