%%%%%%%%%%%%%%%%%%%%%%%%%%%%%%%%
\section{Discussion} \label{S:discussion}
%%%%%%%%%%%%%%%%%%%%%%%%%%%%%%%%

\textcolor{red}{talk about decision statements or thinking-at-loud tasks.}. the listener provide a decision statement on why the selected stimulus sounded more intelligible

We relied on the DAG presented in section \ref{sS:causal_frame} only to define the analyses described \ref{sS:stat_analysis}. A better planning could be done by designing the experimental design according to our hypothesis described by the causal framework. but ....


\begin{comment}
Previous research already used hierarchical models with the replicated entropy measures as outcomes \citep{Boonen_et_al_2021, Faes_et_al_2021}. Hierarchical models are powerful to control for heterogeneity in the data, and also to avoid pre-aggregating procedures that could be pernicious for a proper statistical inference \citep{McElreath_2020}. 

These claims are easier to understand using a though experiment within our research. Consider we have two children with the same mean entropy, but the second child shows more variability across the $10$ utterances than the first. It is clear that the average entropy measure informs about the child's average SI, indicating that both children have similar level. However, the entropy's heterogeneity across the $10$ utterances also informs about the child's SI, as a higher variability imply transcribers agreed less about the second child's intelligibility.

The intuition derived from the previous though experiment is similar to the one presented in \citet{Boonen_et_al_2021}, and it is what justify our use of a hierarchical model.
\end{comment}