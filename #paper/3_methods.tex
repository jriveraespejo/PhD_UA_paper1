%%%%%%%%%%%%%%%%%%%%%%%%%%%%%%%%
\section{Materials and Methods} \label{S:materials_methods}
%%%%%%%%%%%%%%%%%%%%%%%%%%%%%%%%
%
We set up an experiment where speech samples were transcribed by a group of listeners. The current section succinctly describes the participating children, the stimuli used, and the experimental setup, while also delve into the causal and statistical framework of analysis.
%
%
%###############################
\subsection{Children} \label{sS:children}
%###############################
%
Thirty two children were selected using a large corpus of \textit{spontaneously spoken speech}, collected by the Computational Linguistics, Psycholinguistics and Sociolinguistics research center (CLiPS). The selection followed a two step procedure\footnote{similar to one outlined in \citet{Faes_et_al_2021}}. First, a sample of sixteen hearing-impaired children (ten boys, six girls) were selected based on the quality of their registered stimuli. Second, an additional matched sample of sixteen normal hearing children was also selected (six boys, ten girls), and served as a comparison group.
%
\begin{comment}
	Similar designs were used by \citet{Boonen_et_al_2020} and \citet{Faes_et_al_2021}. However, in the former case the number of samples were low, while in the latter, the design was unbalanced and not conducive to appropriate inferences from the pairwise comparisons.
\end{comment}

For the first group, all the hearing-impaired children with cochlear implants (HI/CI) were native speakers of Belgian Dutch, living in Flanders, the Dutch speaking area of Belgium. They were all raised orally using monolingual Dutch, with a limited support of signs. All of the children were screened by the Universal Neonatal Hearing Screening (UNHS), using automated auditory brainstem response hearing tests for newborns, and receive the cochlear implantation before the age of two. Their medical and audiological records did not ascertain any additional health or developmental issues. Hence, no known additional comorbidities were though present. Finally, at the date of the measurement, they were all enrolled in the mainstream educational system.

For the second group, the sixteen normal hearing children (NH) were closely matched to the HI/CI group based on chronological age. All children were also native speakers of Belgian Dutch, and enrolled in the mainstream educational system. None reported hearing loss or additional disabilities, judged from the UNHS screening procedure and their respective parental report.

The characteristics of the selected children is detailed in Table \ref{tab:characteristics}, at the supplementary section \ref{sSA:characteristics}. 
%
%
%###############################
\subsection{Stimuli} \label{sS:stimuli}
%###############################
%
The stimuli consisted of children's utterances, i.e. sentences of similar length, recovered from previously mentioned CLiPS corpus. More specifically, we use a portion of the corpus that consisted of ten utterances recordings for each of the thirty two selected children, adding to a total of $320$ stimuli. 

The stimuli were documented when the child was telling a story cued by the picture book ``Frog, where are you'' \citep{Mayer_1969} to a caregiver ``who does not know the story''. 

The recordings were orthographically transcribed with the CLAN editor in CHAT format \cite{MacWhinney_2020}. The quality of the stimuli was ensured by selecting utterances with no syntactically ill-formed or incomplete sentences, any background noise, cross-talk, long hesitations, revisions or non-words \citep{Boonen_et_al_2021}. The transcriptions were only used in the selection process of the stimuli for the experiment. 
%
\begin{comment}
	under the Design of Experiments (DoE) literature, we would say we have $32$ experimental units with $10$ replicate runs each, making a total of $320$ experimental runs. As it is defined in \citet{Lawson_2015}, an experimental unit is the item under study upon which something is changed, while a replicate run is the experiment conducted with the same factor settings, but using different experimental units. 
\end{comment}
%
%
%###############################
\subsection{Experimental setup} \label{sS:setup}
%###############################
%
The experiment was setup to perform a transcription task in the Qualtrics environment \cite{Qualtrics_2005}. One hundred language students from the University of Antwerp participated. The participants were native speakers of Belgian Dutch, without any particular experience with the speech of hearing-impaired children.

The participants and stimuli were divided into five groups, where each group of $20$ students transcribed $64$ stimuli on their series. The stimuli were presented to the listeners in a random order. As a result, the setup produced $20$ transcriptions per utterance, adding to a total of $6400$ transcriptions. The steps that comprised the task are detailed in the supplementary section \ref{ssSA:transcription}. 
%
\begin{comment}
	present the stimuli based on the \textcolor{blue}{adaptive pairing algorithm} \citep{Pollitt_2012b}?
\end{comment}

The data resulting from the transcription task was then processed and converted into one entropy measure per utterance ($H$), which served as our outcome variable. 

Entropy is a measure bounded in the continuum $[0,1]$, and it was used as a quantification of (dis)agreement between listeners' transcriptions, where utterances yielding a high degree of agreement between transcribers were considered highly intelligible, and therefore registered a lower entropy $\left( H \rightarrow 0 \right)$. In contrast, utterances yielding a low degree of agreement were considered as exhibiting low intelligibility, and therefore registered a higher entropy $\left( H \rightarrow 1 \right)$ \citep{Boonen_et_al_2021, Faes_et_al_2021}. The procedure followed to calculate the entropies is detailed in the supplementary section \ref{ssSA:entropy}.
%
\begin{comment}
	\textbf{for the experimenter:} Based on \citet{Faes_et_al_2021} we depict the procedure for the experimenter:
	%
	\begin{enumerate}
		\item 1. matching procedure 
		\item selection of suitable stimuli
		\item determine the number of stimuli per judge 
	\end{enumerate}	
\end{comment}
%
%
%###############################
\subsection{Causal framework} \label{sS:causal_frame}
%###############################
%
The analysis was informed by a preliminary work aimed at describing the causal and non-causal factors influencing speech intelligibility. More specifically, the current research uses a Directed Acyclic Graph (DAG) \cite{Pearl_2009, Cinelli_et_al_2021} to describe all the relevant variables though to influence intelligibility. A DAG is a type of \textit{structural causal model} that can be represented, among other ways, by a \textit{causal diagram}. 
%
\begin{figure}[h!]
	%
	\centering
	\begin{tikzpicture}
		% nodes
		\node at (2.55,-0.4) {$H^{O}_{bik}$};
		\node at (4,1) {$U_{b}$};
		\node at (4,-1.3) {$U_{ik}$};
		\node at (1,-0.4) {$H^{T}_{i}$};
		\node at (-0.5,-0.4) {$SI_{i}$};
		\node at (-2,-1.3) {$U_{i}$};
		\node at (-0.9,1.75) {$HS_{i}$};
		%\node at (-0.5,2.7) {$U_{HS}$};
		\node at (-2.2,1.5) {$E_{i}$};
		%\node at (-2,2.7) {$U_{E}$};
		\node at (1.5,1.5) {$PTA_{i}$};
		%\node at (1,2.7) {$U_{P}$};
		\node at (-2.1,1) {$A_{i}$};
		%\node at (-2.8,1) {$U_{A}$};
		
		% paths
		\draw[{Circle[open]}-{latex}{Circle}](0.85,0)--(2.7,0); % Hi->Hik
		\draw[{Circle[open]}-{latex}](4,-1)--(2.7,-0.05); % Uk->Hik
		\draw[{Circle[open]}-{latex}](4,0.7)--(2.7,0.05); % Ub->SIi
		\draw[{Circle[open]}-{latex}](-0.5,0)--(0.85,0); % SIi->Hi
		\draw[{Circle[open]}-{latex}](-2,-1)--(-0.5,0); % Ui->SIi
		\draw[{Circle}-{latex}](-0.45,1.55)--(-0.45,0.05); % HSi->SIi
		\draw[-{latex}](-0.45,1.50)--(-1.90,0.75); % HSi->Ai
		\draw[-{latex}](-0.45,1.5)--(0.85,1.5); % HSi->PTAi
		\draw[{Circle}-{latex}](-2,1.5)--(-0.5,1.5); % Ei->HSi
		\draw[-{latex}](-1.9,1.45)--(-0.47,0.05); % Ei->SIi
		\draw[{Circle}-{latex}](0.95,1.55)--(-0.4,0.05); % PTAi->SIi
		\draw[{Circle}-{latex}](-2,0.75)--(-0.5,0); % Ai->SIi
		%\draw[{Circle[open]}-{latex}](-1.95,2.5)--(-1.95,1.6); % UE->Ei
		%\draw[{Circle[open]}-{latex}](-0.45,2.5)--(-0.45,1.6); % UHS->HSi
		%\draw[{Circle[open]}-{latex}](0.95,2.5)--(0.95,1.6); % UP->PTAi
		%\draw[{Circle[open]}-{latex}](-2.8,0.70)--(-2,0.70); % UA->Ai
		
		% extras
		\node at (0.2,-0.25) {$(-)$}; % symbol
		%
		%\draw[dashed] (-2.55,2.1) rectangle (3.2,-2);
		%\node at (-1.5,-1.8) {\small $i=1,\dots,I$};
		%\draw[dotted, thick] (2.1,0.35) rectangle (5.5,-2);
		%\node at (4.5,-1.8) {\small $k=1,\dots,K$};
		%\draw[dash dot, thick] (2.1,-0.7) rectangle (5.5,2.1);
		%\node at (4.5,1.8) {\small $b=1,\dots,B$};
	\end{tikzpicture}
	%
	\caption[DAG: causal diagram]{DAG: causal diagram describing the relationships among the analyzed variables}
	\label{fig:DAG}
\end{figure}

Figure \ref{fig:DAG} shows the \textit{causal diagram} for our research hypothesis. In the figure, $H^{O}_{bik}$ denote the \textit{observed} entropy replicates nested within children and experimental blocks, where $k=1,\dots,10$ utterances, $i=1,\dots,32$ children, and $b=1,\dots,5$ blocks. Moreover, $H^{T}_{i}$ and $SI_{i}$ denotes the child's \textit{true} entropy and speech intelligibility scores, respectively. In addition, $A_{i}$ denotes the children's \textit{hearing age}, $E_{i}$ the etiology of the disease that led to the hearing impairment, $HS_{i}$ the hearing status group, and $PTA_{i}$ the post-implant pure tone average.

Three main features can be emphasized from the figure. First, the children's speech intelligibility and \textit{true} entropy scores are thought to be latent/unobservable variables \cite{Everitt_1984} (drawn with open circles, see supplementary section \ref{sSA:SI} about the appropriate interpretations of the scores). The figure also shows the scores are thought to be inferable from the (observed) entropy replicates. More specifically, we are asserting that the observed entropy replicates $H^{O}_{bik}$ represent multiple realizations of a child's \textit{true} entropy $H^{T}_{i}$. Finally, it shows the intelligibility score $SI_{i}$ is inversely/negatively related to the \textit{true} entropy $H^{T}_{i}$, i.e. the lower the intelligibility the higher the entropy and vice versa, as expected from our the theory. 

Second, the figure reflects the expected hierarchy of variability in our data. This is particularly important as, by failing to account for the appropriate dependencies in the data, we could be ``manufacturing'' false confidence in the parameter's estimates, leading us to incorrect inferences \cite{McElreath_2020}. Based on the experimental setup described in section \ref{sS:setup}, we anticipated the ten utterances, originated from each of the thirty two children, were also observed within a group of transcribers (series) assigned to the observation. Therefore, we expected a hierarchy with children, replicates and block levels ($U_{i}$, $U_{ik}$ and $U_{b}$, respectively).

We expect that if the experiment was ``set up right'', the block random effects would explain a small amount of variability in the data, and its inclusion/exclusion in the model would not change the parameter estimates. Moreover, we expect a larger variability between children's speech intelligibility, at least larger than the block random effects. Several evidence suggest this is particularly true among HI/CI children \cite{Young_et_al_2002, Peng_et_al_2004, Montag_et_al_2014, Castellanos_et_al_2014, Yanbay_et_al_2014, Nittrouer_et_al_2014, Freeman_et_al_2017}. Finally, we did not had any comparable expectation for the variability in the replicates, as this feature has not been investigated before.

Third, the figure shows the assumed relationship among the relevant variables \cite{Niparko_et_al_2010, Boons_et_al_2012, Gillis_2018, Fagan_et_al_2020}, and how these influence the children's intelligibility of speech. Furthermore, it also reveals that we assume the variables are independent, beyond the described relationships. Here follows a description of our causal hypothesis related to the relevant variables.

About \textbf{\textit{hearing age}} and \textbf{\textit{hearing status}}. For the former, we expect it to be one of the main responsible for the increase in speech intelligibility in children. Several studies provide evidence that for NH children, intelligibility increases with chronological age \cite{Chin_et_al_2001, Chin_et_al_2003, Flipsen_2006, Flipsen_2008, Baudonck_et_al_2009, Bowen_2011, Hustad_et_al_2020}, and similar evidence can be found for the HI/CI children. Moreover, recent literature seem to suggest the effects are independent of the children's hearing status \cite{Boonen_et_al_2021}. For the latter, we do not have a clear expectation about the intelligibility levels among the groups. Previous literature suggest that some HI/CI children catch up with their NH counterparts \cite{Wie_2010, Habib_et_al_2010, Boons_et_al_2013, Geers_et_al_2013, Bruijnzeel_et_al_2016, Dettman_et_al_2016, Wie_et_al_2020}. However, other studies also seem to indicate the HI/CI children never reach similar levels than their NH counterparts \cite{Nicholas_et_al_2007, Castellanos_et_al_2014, Chin_et_al_2014, Geers_et_al_2016, Freeman_et_al_2017, Duchesne_et_al_2019, Grandon_et_al_2020}. 

Additionally, we expect \textbf{\textit{pure tone average}} to have a small or null effect on speech intelligibility, as the evidence seem to suggest \cite{Boonen_et_al_2021}. \textit{Pure tone average} is the child's subjective hearing sensitivity, aided or unaided, by their hearing apparatus.

Furthermore, we expect the \textbf{\textit{Etiology}} of the disease, that led to the hearing impairment, to have a differential effect on speech intelligibility, within the HI/CI group. However, since the severity of the etiology cannot be easily ascertain nor ordered, we cannot foresee the direction of such effects, i.e. genetic factors not necessarily lead to worse levels of language development and intelligibility, than factors related to infections.

As expected, it is possible that other unobserved confounding variables are not accounted by our assumptions, and therefore, our causal diagram. This is true for any type of social, behavioral and educational research. However, we argue that the additional transparency of our approach, and its ability to derive statistical procedures from causal assumptions, is its main strength \cite{McElreath_2020, Yarkoni_2020, Rohrer_et_al_2021}.

Finally, we advise the reader to follow the supplementary section \ref{sSA:causal_details} for a extended view of our assumptions and the reasons why other variables, deemed relevant by the literature, were not considered in our hypothesis.
%
%
%###############################
\subsection{Statistical analysis} \label{sS:stat_analysis}
%###############################
%
Using the DAG described in previous section we developed multiple bayesian probabilistic models \cite{Jaynes_2003}. The models reflected the different ways we could state our research hypothesis. Additionally, they were used to validate our statistical models and code.

In essence every model was composed of two parts: (i) a latent measurement model \cite{Everitt_1984}, and (ii) a structural equation model \cite{Hoyle_et_al_2014}. For the former, we represented the child's \textit{speech intelligibility} as a latent variable, inversely related to the child's (latent) \textit{true} entropy. Moreover, we modeled the latter as a variable that can be inferred from the (observed) entropy replicates in the following way:
%
\begin{align}
	%
	H^{O}_{bik} & \sim \text{BetaProp} \left( \bar{H}_{bi}, M_{ik} \right) \\ 
	%
	\bar{H}_{bi} &= \alpha_{b} + H^{T}_{i} \\
	%
	H^{T}_{i} &= \text{logit}^{-1}( -SI_{i} )
	%
\end{align} 

\noindent where $\text{BetaProp}(\mu, \theta)$ defined the beta-proportion distribution with parameters $\mu$ and $\theta$ \cite{Figueroa-Zuniga_et_al_2013, Kruschke_2015}; where $\mu$ is in the unit interval $[0,1]$ and $\theta>0$. Furthermore, $\bar{H}_{bi}$ represented the ``average'' entropy measure nested within blocks ($b=1,\dots,5$) and children ($i=1,\dots,32$), while $M_{ik}$ denoted the ``sample size'' for the beta-proportion distribution, nested within utterances ($k=1,\dots,10$), and possibly nested within children. Furthermore, $H^{O}_{bik}$, $H^{T}_{i}$ and $SI_{i}$ were defined as in the previous section, and $\alpha_{b}$ represented the block random effects. Finally, $\text{logit}^{-1}(x) = \exp(x) / ( 1 + \exp(x) )$ represented the inverse-logit transformation. 

From the previous algebraic structure we can notice. First, we modeled the ``average'' entropy of the replicates with $\mu = \bar{H}_{bi}$, which was composed by the block random effects and the children's \textit{true} entropy. Second, the children \textit{true} entropy $H^{T}_{i}$ is inversely and non-linearly related to the \textit{speech intelligibility} $SI_{i}$. Third and last, we captured the variability present in the entropy replicates around the ``average'' using $\theta = M_{ik}$. See supplementary section \ref{ssSA:model_variability} for a detailed overview of the implications of using this approach.

For the second part, and focus of our research, we used a structural model \cite{Hoyle_et_al_2014}. In this portion, different iterations of our research hypothesis were proposed. Hereby we present the general structural models, from which others were derived:
%
\begin{align}
	%
	SI_{i} & = a_{i} + \alpha + \alpha_{E[i],HS[i]} + \beta_{A, HS[i]} (A_{i} - \bar{A}) + \beta_{P} sPTA_{i} 
	%
\end{align}

\noindent where $A_{i}$, $E_{i}$, $HS_{i}$ are defined as in the previous section, while $sPTA_{i}$ is the standardized version of the post-implant pure tone average $PTA_{i}$. Moreover, $a_{i}$ denoted the children's random intercepts, and $\alpha$ the fixed intercept. On the other hand,  $\alpha_{E[i],HS[i]}$ denoted the fixed effects intercept per etiology and hearing status group, while $\beta_{A, HS[i]}$ denoted the slope of ``hearing age'' per hearing status group. Finally, $\beta_{P}$ described the slope for the standardized pure tone average. 

One important thing need to be noticed from the previous algebraic structure. All the parameters are estimated in the logit scale and centered at $sPTA_{i}=0$ and $\bar{A}$, the latter denoting the minimum hearing age in the sample.

From the previous descriptions, XX models were derived:

%
%
