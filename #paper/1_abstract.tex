\section*{Abstract}

Intelligibility is defined as ``the extent to which a speaker’s message is understood by the listener'' \cite{Munro_et_al_1999}. Its attainment carries an important societal value, as it is a milestone in language development \cite{Chin_et_al_2012}. For its measurement, orthographic transcriptions of utterances are used to construct an entropy score, which express the degree of (dis)agreement in such transcriptions \cite{Boonen_et_al_2021, Shannon_1948}. However, although the benefits of using transcriptions to (indirectly) quantify intelligibility are clear \cite{Boonen_et_al_2020, Boonen_et_al_2021, Hustad_et_al_2020}, the statistical procedures used to model such data are not as sophisticated as the measurement procedures.

Consequently, we propose a novel data analysis using a Bayesian implementation of the Generalized Linear Latent and Mixed Model (GLLAMM) \cite{Rabe_et_al_2004a, Rabe_et_al_2004b, Rabe_et_al_2004c, Rabe_et_al_2012, Skrondal_et_al_2004a}. The statistical procedure offers four benefits. First, it allows to model the bounded entropy data. Second, it `constructs' a speaker's latent intelligibility scale. Third, it test our research hypothesis at the appropriate level. And fourth, it avoids `manufacturing' false confidence in the parameter estimates, producing correct statistical inferences \cite{McElreath_2020}.

As a result we find that, not modeling the bounded nature of the data could lead us to an overestimation of the parameter estimates' precision. For our hypothesis, we see that HI/CI children with genetic etiology have similar levels of intelligibility as NH kids, at `hearing ages' of five. However, children with other etiologies have a significantly lower levels of intelligibility, at same ages. Moreover, we find that NH children develop their intelligibility with each `hearing year' at a higher rate than HI/CI kids, contrary to what was previously found \cite{Boonen_et_al_2021}.
