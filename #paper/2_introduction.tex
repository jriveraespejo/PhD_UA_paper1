%%%%%%%%%%%%%%%%%%%%%%%%%%%%%%%%
\section{Introduction} \label{S:introduction}
%%%%%%%%%%%%%%%%%%%%%%%%%%%%%%%%

Intelligible speech can be defined as the extent to which the elements in an speaker's acoustic signal, e.g. phonemes or words, can be correctly recovered by a listener \citep{Kent_et_al_1989, Whitehill_et_al_2004, vanHeuven_2008, Freeman_et_al_2017}. Because intelligible spoken language requires all core components of speech perception, cognitive processing, linguistic knowledge, and articulation to be mastered \citep{Freeman_et_al_2017}, its attainment carries an important societal value, as it is a milestone in children's language development, the ultimate checkpoint for the success of speech therapy, and has been qualified as the ``gold standard'' for assessing the benefit of cochlear implantation \citep{Chin_et_al_2012}. 

The literature suggest two perspectives from which \textit{speech intelligibility} can be assessed: the message and listener's perspective \citep{Boonen_et_al_2020, Boonen_et_al_2021}. The first, also known as acoustic studies, is focused on assessing separately particular characteristics of the speech samples, e.g. their pitch, duration or stress (supra segmental characteristics), or the articulation of vowels and consonants (segmental characteristics) \citep{Rowe_et_al_2018}. Whereas the second, also known as perceptual studies, is centered on making holistic assessments of the speech stimuli, e.g. measure their perceived quality \citep{Boonen_et_al_2020, Boonen_et_al_2021}. On both instances, the stimuli (children's utterances) can be generated from reading at loud, contextualized utterances, or spontaneous speech tasks\footnote{ordered on increasing level of ecological validity \citep{Flipsen_2006,Ertmer_2011}}.

\begin{comment}
Based on their description, it seems that perceptual are more subjective than acoustic studies, as they do not rely on "objective" measurements, i.e. time duration, wave amplitude, among others, available in the former. However, for the case of SI, there are objective and subjective assessment methodologies.
\end{comment}

Furthermore, perceptual studies can use multiple approaches to measure intelligibility, but they can be largely grouped into two: objective and subjective ratings \citep{Hustad_et_al_2020}. In \textit{objective rating} methods, listeners transcribe the children's utterances orthographically or phonetically, and use such information to construct a score. In that sense, in the transcription task, intelligibility can be inferred from the extent a set of transcribers can identify the word contained in an utterance \cite{Boonen_et_al_2021}. In contrast, under \textit{subjective rating} methods, listeners directly infer the utterance's intelligibility score through specific procedures, e.g. absolute holistic, analytic, or comparative judgments, among others. 

It is easy to deduce that \textit{objective rating} methods produce more valid\footnote{the extent to which scores are appropriate for their intended interpretation and use \citep{Lesterhuis_2018, Trochim_2022}.} and reliable\footnote{the extend to which a measure would give us the same result over and over again \citep{Trochim_2022}, i.e. measure something, free from error, in a consistent way.} scores than its \textit{subjective} counterpart, and as a result, are usually used as an objective measure of intelligibility \citep{Boonen_et_al_2021, Faes_et_al_2021}.

Accompanying the intelligibility assessment methods, the literature supply a myriad of factors that are thought also contribute to the (under)development of intelligible spoken language \cite{Niparko_et_al_2010, Boons_et_al_2012, Gillis_2018, Fagan_et_al_2020}. Among these are audiology related factors, such chronological age, age at implantation, the duration of device use, \textit{hearing age}, bilateral or contralateral cochlear implantation, and the children's preoperative and postoperative hearing levels. On the other hand, there are also child related factors, such as the cause of the hearing impairment (genetic, infections), additional disabilities (mental retardation, speech motor problems), and gender. Finally, there are also environmental factors, such as communication modality. 

Considering all of the above, this paper seeks to investigates the speech intelligibility levels of normal hearing (NH) versus hearing-impaired children with cochlear implants (HI/CI). For that purpose, ten utterances recordings, from thirty two NH and HI/CI children, were selected from a large corpus of \textit{spontaneously spoken speech} collected by the CLiPS research center. Additionally, we set up an experiment, where one hundred language students transcribed each stimuli to the Qualtrics environment \cite{Qualtrics_2005}. Finally, the transcriptions were transformed into an entropy measure per utterance, which served as our outcome variable.

We believe this paper make three specific contributions to the understanding of the factors that drive the intelligibility of spoken language. First, we develop a novel analysis using a latent variable approach \cite{Everitt_1984}. More specifically, we model \textit{speech intelligibility} as a latent variable that can be inferred from the entropy measure replicates. This method offers three specific benefits. On the one hand, the method ``constructs'' an intelligibility score, which in turn, allow us to test different hypothesis and even make individual comparisons at the appropriate level. On the other hand, it allow us to control for different sources of variation. This is particularly important as, by failing to account for the appropriate hierarchies in the data, we could be ``manufacturing'' false confidence in the parameter's estimates, leading us to incorrect inferences \cite{McElreath_2020}. Finally, the method also provides a criterion on how reliable are the entropy replicates to measure speech intelligibility.

Second, we use Directed Acyclic Graph (DAG) \cite{Pearl_2009, Cinelli_et_al_2021} to depict all the relevant variables though to influence speech intelligibility. We describe in detail our causal and non-causal hypothesis, and supplement our description with a causal diagram. The benefit of the method lies, not only, in that it makes the assumptions of our hypothesis more transparent, but also allow us to derive statistical procedures from the aforementioned causal assumptions \cite{McElreath_2020, Yarkoni_2020, Rohrer_et_al_2021}.

Third and final, we wrap the analysis procedure under the Bayesian framework, providing the assumptions, and the steps required to reproduce the computational implementation of the method.
%
%
