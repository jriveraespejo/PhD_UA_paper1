%%%%%%%%%%%%%%%%%%%%%%%%%%%%%%%%
\section{Introduction}
%%%%%%%%%%%%%%%%%%%%%%%%%%%%%%%%

Intelligible speech can be defined as the extent to which the elements in an speaker's acoustic signal, e.g. phonemes or words, can be correctly recovered by a listener \citep{Kent_et_al_1989, Whitehill_et_al_2004, vanHeuven_2008, Freeman_et_al_2017}. Because intelligible spoken language requires all core components of speech perception, cognitive processing, linguistic knowledge, and articulation to be mastered \citep{Freeman_et_al_2017}, its attainment carries an important societal value, as it is a milestone in children's language development, the ultimate checkpoint for the success of speech therapy, and has been qualified as the ``gold standard'' for assessing the benefit of cochlear implantation \citep{Chin_et_al_2012}. 

The literature suggest two perspectives from which \textit{speech intelligibility} can be assessed: the message and listener's perspective \citep{Boonen_et_al_2020, Boonen_et_al_2021}. The first, also known as acoustic studies, is focused on assessing separately particular characteristics of speech samples, e.g. their pitch, duration or stress (supra segmental characteristics), or the articulation of vowels and consonants (segmental characteristics) \citep{Rowe_et_al_2018}. Whereas the second, also known as perceptual studies, is centered on making holistic assessments of the speech stimuli, e.g. measure their perceived quality \citep{Boonen_et_al_2020, Boonen_et_al_2021}. On both instances, the children's utterances can be generated from reading at loud, contextualized utterances, or spontaneous speech tasks\footnote{ordered on increasing level of ecological validity \citep{Flipsen_2006,Ertmer_2011}}.

\begin{comment}
Based on their description, it seems that perceptual are more subjective than acoustic studies, as they do not rely on "objective" measurements, i.e. time duration, wave amplitude, among others, available in the former. However, for the case of SI, there are objective and subjective assessment methodologies.
\end{comment}

Moreover, perceptual studies can use multiple approaches to measure intelligibility, but they can be largely grouped into two: objective and subjective ratings \citep{Hustad_et_al_2020}. In \textit{objective rating} methods, listeners transcribe children's utterances orthographically or phonetically, and use such information to construct a score. In that sense, in the transcription task, intelligibility can be inferred from the extent a set of transcribers can identify the words contained in an utterance \cite{Boonen_et_al_2021}. In contrast, under \textit{subjective rating} methods, listeners directly infer the intelligibility score by assessing the speech sample's quality through specific procedures, e.g. absolute holistic, analytic, or comparative judgments, among others. 

It is easy to infer that \textit{objective rating} methods might produce more valid\footnote{the extent to which scores are appropriate for their intended interpretation and use \citep{Lesterhuis_2018, Trochim_2022}.} and reliable\footnote{the extend to which a measure would give us the same result over and over again \citep{Trochim_2022}, i.e. measure something, free from error, in a consistent way.} scores than the \textit{subjective rating} counterpart, and therefore as their name imply, are usually used as an objective measure of intelligibility \citep{Boonen_et_al_2021, Faes_et_al_2021}.

Considering the previous, this paper investigates the speech intelligibility levels of normal hearing (NH) versus hearing impaired children with cochlear implants (HI/CI). For this purpose, \texcolor{red}{we measured the entropy of representations coming from an spontaneous speech task, resulting from a transcription task.} 

\begin{comment}
add previous evidence about this comparison
\end{comment}

Moreover, the paper make three specific contributions to the measurement and analysis of speech intelligibility. 

\begin{comment}
First, we develop a novel analysis of speech intelligibility using a latent variable approach. More specifically, we model SI as a latent variable inferred from entropy measures coming from the transcription task. This method has XX specific benefits. First, it allows to construct a speech intelligibility score at the children level, which in turn allow us to make comparisons at the individual level. Second, it allow us to control for different types of sources of variation: individual variation (some hypothesis say there is a lot variation in this level), block effects (coming from the experimental design), and measurement error (resulting from the nested structure of measurement), which later allow us to test hypothesis at appropriate levels. Finally, it also measures the reliability (variability) of the transcription task.

Second, we describe in detail the sources of variation in speech intelligibility based on a set of covariates of interest. We supplement our description of the sources of variation with a causal analysis of the factors influencing this variation mention evidence on hearing age, etiology and pta. It is important to understand that a cochlear implant partially restores a severe-to-profound sensorineural hearing loss, i.e. the signal provided by the apparatus is still degraded compared to the signal in normal hearing (Drennan, \& Rubinstein, 2008). However, even under this condition, the device enables children with severe-to-profound hearing impairment to perceive speech and other environmental sounds.

Finally, we wrap the full analysis procedure under the bayesian framework, providing the steps, assumptions and computational implementation of the method.
\end{comment}
