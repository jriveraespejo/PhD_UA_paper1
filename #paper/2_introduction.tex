%%%%%%%%%%%%%%%%%%%%%%%%%%%%%%%%
\section{Introduction} \label{S:introduction}
%%%%%%%%%%%%%%%%%%%%%%%%%%%%%%%%

Intelligible speech can be defined as the extent in which the elements in a speaker's acoustic signal, e.g. phonemes or words, can be correctly recovered by a listener \cite{Freeman_et_al_2017, Kent_et_al_1989, vanHeuven_2008, Whitehill_et_al_2004}. Intelligible spoken language carries an important societal value, as its attainment requires all core components of speech perception, cognitive processing, linguistic knowledge, and articulation to be mastered \cite{Freeman_et_al_2017}. In that sense, \textit{speech intelligibility} is considered a milestone in children's language development, and more practically, it is qualified as the ultimate checkpoint for the success of speech therapy, and the `gold standard' for assessing the benefit of cochlear implantation \cite{Chin_et_al_2012}. 

Multiple approaches can be taken to quantify \textit{speech intelligibility} \cite{Boonen_et_al_2020, Boonen_et_al_2021, Flipsen_2006, Hustad_et_al_2020}, but among them, \textit{objective rating} methods on stimuli recovered from spontaneous speech tasks have received special attention \cite{Boonen_et_al_2021, Hustad_et_al_2020}. In objective rating methods, listeners transcribe children's utterances orthographically (or phonetically), and use such information to construct an intelligibility score. The construction of the score can be done in several ways, e.g. counting the number of (un)intelligible syllables or words in the utterances \cite{Flipsen_2006, Lagerberg_et_al_2014}, or calculating the transcriptions' entropy, a measure that expresses the degree of (dis)agreement in the data \cite{Boonen_et_al_2021, Shannon_1948}. In that sense, the method tries to infer intelligibility from the extent in which a set of transcribers, can identify the words contained in multiple utterances \cite{Boonen_et_al_2021}. 

As the literature suggests, objective rating procedures produce more valid\footnote{validity is understood as the extent to which scores are appropriate for their intended interpretation and use \cite{Lesterhuis_2018, Trochim_2022}.} and reliable\footnote{reliability is though as the extend to which a measure would give us the same result over and over again \cite{Trochim_2022}, i.e. measure something, free from error, in a consistent way.} scores than any other available procedure \cite{Boonen_et_al_2021, Faes_et_al_2021}, as the method does not hinge in the use or production of a \textit{subjective rating scale}, i.e. a scale based on a personal perception of the child's intelligibility. Moreover, the previous advantages are further emphasized by the use of stimuli gathered from spontaneous speech tasks, as they have a greater level of ecological validity, especially compared to contextualized utterances or reading at loud tasks \cite{Flipsen_2006, Ertmer_2011}.

\begin{comment}
	
	The literature suggest two perspectives from which \textit{speech intelligibility} can be assessed: the message and listener's perspective \citep{Boonen_et_al_2020, Boonen_et_al_2021}. The first, also known as acoustic studies, is focused on assessing separately particular characteristics of the speech samples, e.g. their pitch, duration or stress (supra segmental characteristics), or the articulation of vowels and consonants (segmental characteristics) \citep{Rowe_et_al_2018}. Whereas the second, also known as perceptual studies, is centered on making holistic assessments of the speech stimuli, e.g. measure their perceived quality \citep{Boonen_et_al_2020, Boonen_et_al_2021}. On both instances, the stimuli (children's utterances) can be generated from reading at loud, contextualized utterances, or spontaneous speech tasks\footnote{ordered on increasing level of ecological validity \cite{Flipsen_2006, Ertmer_2011}}.
	
	%%%%%%%%%%%%%%%%
	Based on their description, it seems that perceptual are more subjective than acoustic studies, as they do not rely on "objective" measurements, i.e. time duration, wave amplitude, among others, available in the former. However, for the case of SI, there are objective and subjective assessment methodologies.
	%%%%%%%%%%%%%%%%
	
	Furthermore, perceptual studies can use multiple approaches to measure intelligibility, but they can be largely grouped into two: objective and subjective ratings \citep{Hustad_et_al_2020}. In \textit{objective rating} methods, listeners transcribe the children's utterances orthographically or phonetically, and use such information to construct a score. In that sense, in the transcription task, intelligibility can be inferred from the extent a set of transcribers can identify the word contained in an utterance \cite{Boonen_et_al_2021}. In contrast, under \textit{subjective rating} methods, listeners directly infer the utterance's intelligibility score through specific procedures, e.g. absolute holistic, analytic, or comparative judgments, among others. 
	
	It is easy to deduce that \textit{objective rating} methods produce more valid\footnote{the extent to which scores are appropriate for their intended interpretation and use \citep{Lesterhuis_2018, Trochim_2022}.} and reliable\footnote{the extend to which a measure would give us the same result over and over again \citep{Trochim_2022}, i.e. measure something, free from error, in a consistent way.} scores than its \textit{subjective} counterpart, and as a result, are usually used as an objective measure of intelligibility \citep{Boonen_et_al_2021, Faes_et_al_2021}.
	
\end{comment}

However, although the literature is clear on the method's benefits to measure \textit{speech intelligibility} \cite{Boonen_et_al_2020, Boonen_et_al_2021, Hustad_et_al_2020}, we notice the statistical approaches used to model such data still face three important issues, and these come to the detriment of the measurement procedure's sophistication. 

First, as previous paragraphs reveal, the intelligibility scores are `complex' in nature, however, such `complexity' is rarely fully considered in the statistical modeling procedure. The problem with the later is that, because the data does not fulfill the typical assumptions, e.g. normality, its analysis under such models might lead us to erroneous conclusions \textcolor{red}{[citation]}. On the one hand, outcomes such as the number of (un)intelligible words are discrete, while the entropy scores are continuous in nature. In addition, there is the consideration that both measures are constraint in specific bounds, i.e. the number of (un)intelligible words cannot be negative, while the entropy scores are in the bounds between zero and one. Finally, given the measurement procedure's nature, the scores are produced in a clustered manner, i.e. we observe several score measurement per child. Considering all of the above, it is clear the modeling requires some adjustments to account for all of these nuances, in favor of proper statistical inferences.

So far the literature shows the applied statistical procedures have always assumed `normality', examples of this can be seen in \citet{Boonen_et_al_2021, Flipsen_et_al_2006} and \citet{Hustad_et_al_2020}. In addition, some papers in the literature have even used multilevel modeling to deal with the clustered nature of the data, an example of this can be found in \citet{Boonen_et_al_2021}. However, to the authors knowledge, no paper have dealt with all of the data `complexity' at once, which leads us to believe that, by using more sophisticated statistical models we could improve our statistical inferences. 

Second, although the literature suggest the number of (un)intelligible words or the entropy of transcriptions are scores that captures the level of intelligibility in a child, it is easy to notice that these two can still be considered surrogate measures of it, i.e. scores that indirectly reflect what is intended to be measured. The latter is important because it implies the scores (outcomes) are `measured with error', indicating there is an unobserved `construct' that is responsible for the variation observed in them, i.e. the \textit{speech intelligibility} of a child. Moreover, is important to recognize that this `error' is of a different kind that the one produced by the clustered nature of the data, and that by failing to account for it, would also lead us to produce incorrect inferences \cite{deHaan_et_al_2019}.

To the authors knowledge, no attempt on constructing an actual intelligibility scale have been made. Therefore, we believe the literature could benefit from showing how to implement such models, in combination with all the statistical procedures needed to account for all the aforementioned nuances of the data.

Third, even though the literature supplies a myriad of factors that are thought to contribute to the (under)development of intelligible spoken language \cite{Boons_et_al_2012, Gillis_2018, Fagan_et_al_2020, Niparko_et_al_2010}, no unified framework of analysis is used to determine which factors are relevant, or conforms to valid and actionable causal hypothesis. This lack of framework not only makes the selection of relevant factors harder, but also hinders the researcher's ability to determine which factors can be analyzed in tandem without facing some common statistical issues, e.g. including two variables to the model that provide similar type of information \cite{Boonen_et_al_2021}, which we know could cause multicollinearity \cite{Farrar_et_al_1967}, ultimately affecting the inference capabilities of the model. 

The factors that are proposed by the literature can be grouped into audiology, child and environmental related factors. For the first, there is chronological age, age at implantation, the duration of device use, `hearing' age, bilateral or contralateral cochlear implantation, and the children's preoperative and postoperative hearing levels. For the second, there is the cause of the hearing impairment or etiology (genetic, infections), additional disabilities (mental retardation, speech motor problems), and gender. Finally for the last, there is communication modality.  

Therefore, considering all of the aforementioned variables, and the relations' complexities among themselves and with the outcome, we believe that proposing a causal framework of analysis would provide a more transparent way of stating the hypothesis of our research.

Considering all of the above, we believe this paper make three specific contributions to the field. First, we develop a novel analysis using a Generalized Linear Latent and Mixed Model (GLLAMM) \cite{Rabe_et_al_2004a, Rabe_et_al_2004b, Rabe_et_al_2004c, Rabe_et_al_2012, Skrondal_et_al_2004a}. More specifically, we model \textit{speech intelligibility} as a latent variable \cite{Everitt_1984} that can be inferred from the entropy replicates, which in turn are modeled under a Generalized Linear Mixed Model (GLMM) \cite{Breslow_et_al_1993, Nelder_et_al_1996, Nelder_et_al_1983}. This method offers three specific benefits. On the one hand, the method `constructs' an intelligibility score, which in turn, allow us to test different hypothesis and even make individual comparisons at the children level. On the other hand, it allow us to control for different sources of variation. This is particularly important as, by failing to account for the appropriate hierarchies in the data, we could be `manufacturing' false confidence in the parameter estimates, leading us to incorrect inferences \cite{McElreath_2020}. Finally, the method also provides a `criterion' on how reliable are the entropy replicates to measure speech intelligibility.

Second, we use Directed Acyclic Graph (DAG) \cite{Pearl_2009, Cinelli_et_al_2021} to depict all the relevant variables though to influence speech intelligibility. We describe in detail our causal and non-causal hypothesis, and supplement our description with a causal diagram. The benefit of the method lies, not only, in that it makes the assumptions of our hypothesis more transparent, but also allow us to derive statistical procedures from the aforementioned causal assumptions \cite{McElreath_2020, Yarkoni_2020, Rohrer_et_al_2021}.


\begin{comment}
Third, we wrap the analysis procedure under the Bayesian framework, providing the assumptions, and the steps required to reproduce the computational implementation of the models. Hypothesis seem clear cut with classical statistics, but bayesian modeling allow us to have a more nuanced view of the models, their inferences and conclusions

Considering all of the above, this paper seeks to investigates the speech intelligibility levels of normal hearing (NH) versus hearing-impaired children with cochlear implants (HI/CI). For that purpose, ten utterances recordings, from thirty two NH and HI/CI children, were selected from a large corpus of \textit{spontaneously spoken speech} collected by the CLiPS research center. Additionally, we set up an experiment, where one hundred language students transcribed each stimuli to the Qualtrics environment \cite{Qualtrics_2005}. Finally, the transcriptions were transformed into an entropy measure per utterance, which served as our outcome variable.
\end{comment}
%
%
