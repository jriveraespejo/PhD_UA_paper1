%%%%%%%%%%%%%%%%%%%%%%%%%%%%%%%%
\section{Introduction} \label{S:introduction}
%%%%%%%%%%%%%%%%%%%%%%%%%%%%%%%%

\textit{Intelligible} spoken language requires all core components of speech perception, cognitive processing, linguistic knowledge, and articulation to be mastered \cite{Freeman_et_al_2017}. In that sense, its attainment carries an important societal value, as it is considered a milestone in children's language development; and more practically, it is qualified as the ultimate checkpoint for the success of speech therapy, and the `gold standard' for assessing the benefit of cochlear implantation \cite{Chin_et_al_2012}.

But what is speech \textit{intelligibility}?. \textit{Intelligibility} can be broadly defined as ``the extent to which a speaker's message is actually understood by the listener" \cite{Munro_et_al_1999}. But in a more narrow sense, it refers to the listener’s ability to successfully identify (decode) the words in a message \cite{Freeman_et_al_2017, Kent_et_al_1989, vanHeuven_2008, Whitehill_et_al_2004}. The latter definition is more helpful, as it sets a clear contrast with comprehensibility, which involves the listener’s ability to understand the message, and its intent \cite{Munro_et_al_1999, Smith_et_al_1985}.

However, indifferent of its broad or narrow definition, the literature reveal that intelligibility can be further compromised by features of the communicative environment, such as noise \cite{Munro_1998}; by features of the speaker, like speaking rate \cite{Munro_et_al_1998} or accent \cite{Jenkins_2000, Ockey_et_al_2016}}; or features of the listener, like vocabulary mastery \cite{Varonis_et_al_1985}. Moreover, the latter emphasizes its highly dynamical nature, where changes in intelligibility stem from on line adaptations of the speaker, to the listener and/or the context.

Therefore, we can say that speech intelligibility generate considerable interest for its societal value, but its measurement pose interesting challenges, particularly because of its entanglement with other features of the communication.

Considering the previous, the literature suggest two perspectives from which intelligibility can be assessed: the message and listener’s perspective \cite{Boonen_et_al_2020, Boonen_et_al_2021}. The first, also known as acoustic studies, is focused on assessing separately particular characteristics of the speech samples, e.g. their pitch, duration, stress, or the articulation of vowels and consonants \cite{Rowe_et_al_2018}. Whereas the second, also known as perceptual studies, is centered on making holistic assessments of the speech stimuli, e.g. measure their overall quality \cite{Boonen_et_al_2020, Boonen_et_al_2021}. The former is justified by the fact that using the speech samples, we can detect articulatory, acoustic, and auditory characteristics of intelligible utterances. In contrast, the latter is justified by the notion that intelligibility is a concept `that everyone can judge', but can only be measured indirectly \cite{Guilford_1954, Stevens_1946}.

\begin{comment}
	
	On both instances, the stimuli (children's utterances) can be generated from reading at loud, contextualized utterances, or spontaneous speech tasks\footnote{ordered on increasing level of ecological validity \cite{Flipsen_2006, Ertmer_2011}}.
	
	%%%%%%%%%%%%%%%%
	Based on their description, it seems that perceptual are more subjective than acoustic studies, as they do not rely on "objective" measurements, i.e. time duration, wave amplitude, among others, available in the former. However, for the case of SI, there are objective and subjective assessment methodologies.
	%%%%%%%%%%%%%%%%
		
\end{comment}

Focusing our attention on perceptual studies, `objective' rating methods on children's utterances recovered from spontaneous speech tasks, have received special attention \cite{Boonen_et_al_2021, Hustad_et_al_2020}. In these methods, listeners transcribe children's utterances orthographically (or phonetically), and use these transcriptions as information to construct an intelligibility score; more precisely, an entropy score that expresses the degree of (dis)agreement in the transcriptions \cite{Boonen_et_al_2021, Shannon_1948}. As a result, we obtain scores that are clustered and bounded in nature. Clustered because we get multiple measurements per child (one per utterance), and bounded because their values are in the continuum between zero and one.

\begin{comment}
	
	As the literature suggests, objective rating procedures produce more valid\footnote{validity is understood as the extent to which scores are appropriate for their intended interpretation and use \cite{Lesterhuis_2018, Trochim_2022}.} and reliable\footnote{reliability is though as the extend to which a measure would give us the same result over and over again \cite{Trochim_2022}, i.e. measure something, free from error, in a consistent way.} scores than any other available procedure \cite{Boonen_et_al_2021, Faes_et_al_2021}, as the method does not hinge in the use or production of a \textit{subjective rating scale}, i.e. a scale based on a personal perception of the child's intelligibility. Moreover, the previous advantages are further emphasized by the use of stimuli gathered from spontaneous speech tasks, as they have a greater level of ecological validity, especially compared to contextualized utterances or reading at loud tasks \cite{Flipsen_2006, Ertmer_2011}.
	
\end{comment}

Therefore, `objective' rating methods try to infer intelligibility from the extent in which a set of transcribers can identify the words contained in the utterances \cite{Boonen_et_al_2021}. In other words, we get a `proxy' measure of the speaker's intelligibility as judged by a listener, a snapshot of his/her performance under a specific set of circumstances \cite{Hustad_et_al_2020}. Moreover, the epistemological certainty in such ‘snapshot’ as a measure of intelligibility stems from the design and steps taken to collect the data.

However, although the literature is clear on the benefits of `objective' rating methods to (indirectly) quantify intelligibility \cite{Boonen_et_al_2020, Boonen_et_al_2021, Hustad_et_al_2020}, we notice the statistical procedures used to model such data are not at par of the measurement procedure's sophistication.

Previous research have considered the clustered nature of the data but ignored its bounded nature, where averaging was considered a `valid' option for modeling \cite{Boonen_et_al_2021}. We argue that the latter practice is not appropriate, as with bounded data not only the location (average), but also the spread (variance) of the distribution, might inform about the speaker's intelligibility \cite{Nelder_et_al_1983}.

\begin{comment}
	
	First, as previous paragraphs reveal, the intelligibility scores are `complex' in nature, however, such `complexity' is rarely fully considered in the statistical modeling procedure. The problem with the later is that, because the data does not fulfill the typical assumptions, e.g. normality, its analysis under such models might lead us to erroneous conclusions \textcolor{red}{[citation]}. On the one hand, outcomes such as the number of (un)intelligible words are discrete, while the entropy scores are continuous in nature. In addition, there is the consideration that both scores are constraint in specific bounds, i.e. the number of (un)intelligible words cannot be negative, while the entropy scores are in the bounds between zero and one. Finally, given the rating procedure's nature, the scores are produced in a clustered manner, i.e. we observe several score measurements per child. 
	
	So far the literature shows that, even when the data does not conform to the `normality' assumption, the applied statistical procedures are still supported on it, examples of this can be seen in \citet{Boonen_et_al_2021, Flipsen_et_al_2006} and \citet{Hustad_et_al_2020}. In addition, some papers in the literature have even used (hierarchical) multilevel modeling to deal with the clustered nature of the data, e.g. \citet{Boonen_et_al_2021}. However, to the authors knowledge, no paper have dealt with all of the data nuances at once, which leads us to believe that, by using more sophisticated statistical models we could improve our statistical inferences. 

\end{comment}

Furthermore, in order to understand or intervene on the factors that drives speech intelligibility, first one needs to `construct an error free' \textit{intelligibility} scale \cite{Carroll_2006}, a characteristic not possessed by the entropy measures nor its averages.

\begin{comment}
	
	Second, although the literature suggest the number of (un)intelligible words or the entropy of transcriptions are scores that capture the level of intelligibility in a child, it is easy to notice these two can still be considered surrogate measures of it, i.e. scores that indirectly reflect what is intended to be measured. The latter is important because it implies these outcomes are `measured with error', resulting from considering that there is an unobserved (latent) `construct' that is responsible for the observed scores variation, i.e. the \textit{speech intelligibility}. Moreover, it is important to recognize that this `measurement error' is of a different kind that the one produced by the clustered nature of the data, and that again, by failing to account for it, we would be led to incorrect inferences \cite{deHaan_et_al_2019}.
	
	To the authors knowledge, no attempt to create such intelligibility latent 'construct' have been made. Therefore, we believe the literature could benefit from showing how to implement such procedure in a statistical model, in combination with the procedures needed to account for the other nuances in the data. 
	
	Third, even though the literature supplies a myriad of factors that are thought to contribute to the (under)development of intelligible spoken language \cite{Boons_et_al_2012, Gillis_2018, Fagan_et_al_2020, Niparko_et_al_2010}, no transparent framework of analysis is used to determine which factors are relevant, or conforms to valid and actionable causal hypothesis. The lack of such framework not only makes the selection and assessment of relevant factors harder, but also hinders the researcher's ability to avoid facing some common statistical issues related to such selection, e.g. determine which factors can be analyzed in tandem without facing collinearity problems, which ultimately affects our inference capabilities \cite{Farrar_et_al_1967}.
	
	As it was suggested, several factors are proposed by the literature, but these can be largely grouped into three categories: audiology, child and environmental related factors. For the first, they are the chronological age, age at implantation, the duration of device use, `hearing' age, bilateral or contralateral cochlear implantation, and the children's preoperative and postoperative hearing levels. For the second, there is the etiology or the cause of the hearing impairment (e.g. genetic, infections), additional disabilities (e.g. mental retardation, speech motor problems), and gender. Finally for the last, there is the communication modality. 
	
	Therefore, considering the aforementioned variables, and the relation complexity with themselves and the outcome, we believe that a causal framework would allow us to integrate previous literature on the matter, and also provide a more transparent way of state and analyze our research hypothesis.
	
\end{comment}

Considering all of the above, we propose a novel analysis of the entropy data using a Bayesian implementation of the Generalized Linear Latent and Mixed Model (GLLAMM) \cite{Rabe_et_al_2004a, Rabe_et_al_2004b, Rabe_et_al_2004c, Rabe_et_al_2012, Skrondal_et_al_2004a}. The statistical procedure offers four benefits. First, it allows to appropriately model the bounded entropy data. Second, it provides a way to `construct' the speaker's latent intelligibility scale. Third, it allow us to test our research hypothesis at the appropriate level. And fourth, as a result from the first two, we successfully avoid producing false confidence in the parameter estimates, which help us to produce correct statistical inferences \cite{McElreath_2020}.

\begin{comment}
	
	The previous statistical method offers three specific benefits. On the one hand, it allow us to consider all of the data nuances at once, i.e. we can model our `non normal' data, and control for the different sources of variation (error) observed in it. The latter is particularly important because, as it was mentioned, by failing to account for these sources we could be `manufacturing' false confidence in the parameter estimates, leading us to incorrect inferences \cite{McElreath_2020}. On the other hand, the method provides a way to `construct' an intelligibility scale. This in turn, allow us to test our research hypotheses on the measure of interest, and even make individual comparisons at the children level. Finally, resulting from the statistical procedure sophistication, the method also provides a `criterion' on how reliable the repeated entropy measures are to quantify speech intelligibility.
	
	Second, we use Directed Acyclic Graph (DAG) \cite{Pearl_2009, Cinelli_et_al_2021} to depict all the relevant variables though to influence \textit{speech intelligibility}. We describe in detail our causal and non-causal hypothesis, and supplement our description with a causal diagram. The benefit of the method lies, not only, in that it makes the assumptions of our hypothesis more transparent, but also allow us to derive statistical procedures from our causal assumptions \cite{McElreath_2020, Yarkoni_2020, Rohrer_et_al_2021}.
	
	Third, given the complexity of the statistical procedure, we wrap the analysis under the Bayesian framework, providing the assumptions and steps required to reproduce the computational implementation of the models. The general reasons for using Bayesian statistics in our research are that the framework can handle all kinds of data-generating processes \cite{Fox_2010}, and it lends itself easily to complex and over-parameterized models \cite{Baker_1998, Kim_1999}, characteristics that define our implementation. Furthermore, although the framework have similar estimation capabilities as its frequentist counterpart \cite{Baker_1998, Hsieh_2010, Wollack_2002}, some specific scenarios in our current research also favors its use, i.e. the need of inferences with a small sample size \cite{Fox_2010, McElreath_2020, Skrondal_et_al_2004a}, and the need of confining some parameters in a permitted space \cite{Martin_et_al_1975}, e.g. variances confined to positive values. Moreover, since the main output of Bayesian statistics are not point estimates, but rather the posterior distribution of the parameters' possible values \cite{McElreath_2020}, the framework allow us to have a more nuanced view of our inferences and conclusions.

\end{comment}
	
We find the proposed method bring new insights about the use of replicated entropy scores to measure intelligibility, and on how some factors affect the (under)development of children's intelligibility.

\begin{comment}
	
	Fourth, we implement all of the above in a data set consisting of repeated entropy measures, with the purpose of determine which factors affect the \textit{speech intelligibility} levels of normal hearing (NH) versus hearing-impaired children with cochlear implants (HI/CI). The entropy measures were calculated using the transcriptions of one hundred language students from the University of Antwerp, where each student transcribed the stimuli to the Qualtrics environment \cite{Qualtrics_2005}. The stimuli consisted in ten utterances recordings for each of the thirty two NH and HI/CI children, selected from a large corpus of \textit{spontaneously spoken speech} collected by the Computational Linguistic and Psycholinguistics Research Centre (CLiPS).
	
\end{comment}

On the one hand, the method reveal that, not integrating the bounded nature of the data in the modeling procedure could lead us to wrongful statistical conclusions. More precisely, it could lead us to an overestimation of the parameter estimates' precision.

Lastly, our hypothesis tests reveal that hearing impaired children with cochlear implants (HI/CI) and genetic etiology have similar levels of intelligibility as normal hearing kids (NH), when both groups have a `hearing ages' of five. However, the same cannot be said for children with other etiologies, like CMV infection or other causes, as they start a significantly lower level of intelligibility at same `ages'. Moreover, our tests found enough evidence to assert that NH children develop their intelligibility with each `hearing year' at a higher rate than HI/CI kids. This offer evidence contrary to what was previously found \cite{Boonen_et_al_2021}.

Finally, we observe our results support the hypothesis that HI/CI children with severe hearing loss, as accounted by the pure tone average, develop their language at a slower rate than their NH counterparts.
%
%
