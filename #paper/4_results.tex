%%%%%%%%%%%%%%%%%%%%%%%%%%%%%%%%
\section{Results} \label{S:results}
%%%%%%%%%%%%%%%%%%%%%%%%%%%%%%%%
%
%###############################
\subsection{About our hypothesis} \label{sS:results_hypothesis}
%###############################
%
The current research used the Information-Theoretic Approach \citep{Anderson_2008, Chamberlain_1965} for model selection and inference. The application of the approach required: (i) the expression of the research hypothesis into statistical models, (ii) the selection of the most plausible models, and (iii) to produce inferences based on one or multiple selected models. The first requirement of the approach is covered in sections \ref{sS:causal_frame} and \ref{sS:stat_analysis}, and expanded in supplementary section \ref{sSA:model_details}. Here we use the results of the second requirement, detailed in the supplementary section \ref{ssSA:model_selection}, to produce the final inferences. 

As detailed in the supplementary section, the final conclusions of our research will be drawn from the comparisons of two models: (i) a no interaction model, with one estimated `sample size' (model 3), and (ii) a full interaction model, with one estimated `sample size' (model 10). The former is selected as it is the model with highest probabilistic support. The latter is considered because it encompasses the remaining highest supported models. Furthermore, no \textit{robust} model is inspected, as we prefer a more parsimonious depiction of our hypothesis. Table \ref{tab:results} summarizes the parameters posterior estimates and contrasts of interest.
%
\begin{table}[h!]
	\centering
	\begin{tabular}{|cccccccccccc|} 
		\hline
		& \multicolumn{3}{c}{} & \multicolumn{2}{c}{CI} & & \multicolumn{2}{c}{HPDI} & & \multicolumn{2}{c|}{}\\[0.5ex]
		\cline{5-6} \cline{8-9}
		Parameter & Mean & SD & & $2.5\%$ & $97.5\%$ & & $2.5\%$ & $97.5\%$ & & n eff. & Rhat \\[0.5ex] 
		\hline\hline
		\rowcolor{gray}
		\multicolumn{12}{|l|}{ \textbf{Model 3: No interaction (one `size')} } \\
		\multicolumn{12}{|l|}{ \textbf{parameters:} } \\
		\texttt{a} & 0.179 & 0.189 & & -0.197 & 0.554 & & -0.198 & 0.552 & & 3677.705 & 1.001\\
		\texttt{bP[2]} & -0.117 & 0.166 & & -0.439 & 0.207 & & -0.440 & 0.202 & & 1738.855 & 1.000 \\
		\texttt{bA} & 0.432 & 0.141 & & 0.154 & 0.715 & & 0.170 & 0.723 & & 1815.243 & 1.001 \\
		\texttt{aHS[1]} & 0.284 & 0.235 & & -0.178 & 0.761 & & -0.203 & 0.728 & & 2719.833 & 1.000 \\
		\texttt{aHS[2]} & 0.116 & 0.217 & & -0.303 & 0.537 & & -0.304 & 0.537 & & 2646.671 & 1.000 \\
		\multicolumn{12}{|l|}{ \textbf{constrast:} } \\
		\texttt{aHS[2]-aHS[1]} & -0.168 & 0.246 & & -0.661 & 0.371 & & -0.650 & 0.324 & & n.a. & n.a. \\
		\multicolumn{12}{|l|}{ } \\
		\rowcolor{gray}
		\multicolumn{12}{|l|}{ \textbf{Model 10: Full interaction (one `size')} } \\
		\multicolumn{12}{|l|}{ \textbf{parameters:} } \\
		\texttt{a} & 0.217 & 0.179 & & -0.142 & 0.562 & & -0.118 & 0.580 & & 3902.629 & 0.999 \\
		\texttt{bP[2]} & -0.122 & 0.173 & & -0.457 & 0.220 & & -0.460 & 0.216 & & 1659.639 & 1.002 \\
		\texttt{bAHS[1]} & 0.435 & 0.157 & & 0.127 & 0.745 & & 0.123 & 0.741 & & 1477.460 & 1.000 \\
		\texttt{bAHS[2]} & 0.237 & 0.177 & & -0.119 & 0.596 & & -0.089 & 0.615 & & 1844.785 & 1.001 \\
		\texttt{aEHS[1,1]} & 0.183 & 0.278 & & -0.358 & 0.726 & & -0.349 & 0.729 & & 2524.451 & 1.000 \\
		\texttt{aEHS[2,2]} & 0.212 & 0.241 & & -0.259 & 0.684 & & -0.259 & 0.684 & & 2418.275 & 0.999 \\
		\texttt{aEHS[3,2]} & 0.077 & 0.245 & & -0.402 & 0.547 & & -0.383 & 0.561 & & 3015.559 & 0.999 \\
		\texttt{aEHS[4,2]} & 0.007 & 0.269 & & -0.522 & 0.530 & & -0.547 & 0.499 & & 3268.043 & 1.000 \\
		\multicolumn{12}{|l|}{ \textbf{contrast:} } \\
		\texttt{bAHS[2]-bAHS[1]} & -0.197 & 0.208 & & -0.600 & 0.213 & & -0.590 & 0.223 & & n.a. & n.a. \\
		\texttt{aEHS[2,2]-aEHS[1,1]} & 0.029 & 0.352 & & -0.673 & 0.723 & & -0.714 & 0.682 & & n.a. & n.a. \\
		\texttt{aEHS[3,2]-aEHS[1,1]} & -0.106 & 0.365 & & -0.823 & 0.607 & & -0.822 & 0.609 & & n.a. & n.a. \\
		\texttt{aEHS[4,2]-aEHS[1,1]} & -0.176 & 0.380 & & -0.906 & 0.568 & & -0.928 & 0.535 & & n.a. & n.a. \\
		\texttt{aEHS[3,2]-aEHS[2,2]} & -0.135 & 0.299 & & -0.719 & 0.453 & & -0.730 & 0.439 & & n.a. & n.a. \\
		\texttt{aEHS[4,2]-aEHS[2,2]} & -0.205 & 0.344 & & -0.888 & 0.453 & & -0.902 & 0.430 & & n.a. & n.a. \\
		\texttt{aEHS[4,2]-aEHS[3,2]} & -0.070 & 0.344 & & -0.744 & 0.612 & & -0.735 & 0.617 & & n.a. & n.a. \\
		\hline
		\multicolumn{12}{l}{\footnotesize{CI = compatibility interval}} \\
		\multicolumn{12}{l}{\footnotesize{HDPI = highest posterior density interval}} \\
		\multicolumn{12}{l}{\footnotesize{n eff. = effective number samples}} \\
		\multicolumn{12}{l}{\footnotesize{Rhat = Gelman-Rubin diagnostic}} \\
		\multicolumn{12}{l}{\footnotesize{n.a. = not available / not applicable}} \\
		\multicolumn{12}{l}{\footnotesize{[1] = NH children, [2] = HI/CI children}} \\
		\multicolumn{12}{l}{\footnotesize{[1,i] = NH children, [2,i] = genetic, [3,i] = CMV infection, [4,i] = unknown etiology}} \\
	\end{tabular}
	\caption[Selected statistical models: results]{Selected statistical models: results.}
	\label{tab:results}
\end{table}

Before providing any parameter interpretation, it is important to highlight an statistical issue that will permeate all of our inferences. Given the large amount of variability registered at the children and replicates levels, the models will not be able to produce unequivocal null hypothesis rejections, i.e. CI and HPDI not including the zero value (see section \ref{sS:results_variability}). However, from supplementary section \ref{ssSA:model_simulation}, we can be certain that our models can affirm the estimate values with at least $60\%$ power, depending on the size of the estimates; i.e. our models can correctly reject the null hypothesis, when a specific alternative hypothesis is true. Therefore we will continue interpreting the parameters, but the reader should cautious into understanding the CI and HPDI indicate that our data size might still not be enough, to ultimately define the direction of the effects, and further evidence might still be needed.

First, about \textbf{\textit{hearing status}}, model three reveals that HI/CI children have a modest lower level of \textit{speech intelligibility}, compared to their NH counterparts (\texttt{aHS[2]-aHS[1]}). The result then seem to support previous evidence on the matter \cite{Nicholas_et_al_2007, Castellanos_et_al_2014, Chin_et_al_2014, Geers_et_al_2016, Freeman_et_al_2017, Duchesne_et_al_2019, Grandon_et_al_2020}. However, model ten paints a more nuanced story considering the \textbf{\textit{Etiology}} of the disease. The contrasts of the model reveal that HI/CI children with genetic etiology manage to reach similar levels of \textit{intelligibility} as NH children (\texttt{aEHS[2,2]-aEHS[1,1]}). Nevertheless, the same cannot be said when the hearing impairment is caused either by a CMV infection (\texttt{aEHS[3,2]-aEHS[1,1]}) or other unknown causes (\texttt{aEHS[4,2]-aEHS[1,1]}).

About the former result, we theorize a intuitive explanation. As children with genetic causes have grown developing their language using the hearing apparatus, they grew developing the appropriate neural connections to benefit from the new hearing input from the start, even when that input is slightly degraded \cite{Drennan_et_al_2008}. However, the same cannot be said for other etiology status. What is hinted is that, a child with other etiology status might need to `rewire' his(her) neural connection in order to fully take advantage of the new hearing signal, as it is assumed they begin their life hearing normally, and then loosing the ability to process sound. This result is particularly important, as it might reveal the factors that caused the hearing impairment matter, even when the children receive the new hearing input signal at a young age.

In relation to \textbf{\textit{hearing age}}, model three reveals what we initially expected: the factor is one of the main determinants for the increase in children's \textit{speech intelligibility}, showing moderate effects (\texttt{bA}) \cite{Cohen_1988, Sawilowsky_2009}. However, model ten indicates there seem to be a difference in the evolution of \textit{intelligibility} between HI/CI and NH children. The table shows the evolution per unit of \textit{hearing age} above five years in NH children is around $0.43$ logits (\texttt{bAHS[1]}), while the evolution for HI/CI children is around $0.24$ logits (\texttt{bAHS[2]}), approximately $0.19$ lower than their NH analogues (\texttt{bAHS[2]-bAHS[1]}). This result provides preliminary evidence contrary to what was previously found \cite{Boonen_et_al_2021}, indicating that HI/CI develop their language and intelligibility at a slower rate.

Finally, for \textbf{\textit{pure tone average}}, we see both models also support our initial hypothesis: HI/CI children with severe hearing loss, as accounted by the variable, develop their language at a slower rate than their NH counterparts (\texttt{bP[2]}). It is good to point out that, from the perspective of effect sizes the estimates can be considered as small \cite{Cohen_1988, Sawilowsky_2009}, but present nonetheless (as proved by our power analysis).
%
%
\begin{figure}[!h]
	\centering
	\includegraphics[width=0.55\linewidth]{variability_plot.pdf}
	\caption[Posterior predictive: hierarchy of variability in the data]{Posterior predictive: hierarchy of variability in the data. Distributions are plotted at the entropy replicates scale, considering three different average entropy values: $\mu=0.2$, $\mu=0.5$, and $\mu=0.8$ (discontinuous lines). Thick solid lines represent the marginal distribution, thin solid lines depict $100$ random posterior samples.}
	\label{fig:variability}
\end{figure}
%
%
%###############################
\subsection{About the hierarchy of variability} \label{sS:results_variability}
%###############################
%
As expected, three hierarchies of variability were present in our data: the blocks and children's random effects, as well as, the variability of the entropy replicates.

Evidence from the posterior estimates reveal the block random effects explained a small amount of variability in the data (top panel of figure \ref{fig:variability}), and its inclusion/exclusion in the model did not change the parameter estimates. The previous implies the experiment was correctly `set up', as the series in which the utterances were transcribed did not explain a significant amount of variation, nor its exclusion biased the parameter estimates.

On the contrary, we observe a significantly larger variability between children's \textit{speech intelligibility}, more precisely, more than three times the block effects variability (middle panel of figure \ref{fig:variability}). The previous corroborated preliminary evidence on the matter \cite{Young_et_al_2002, Peng_et_al_2004, Montag_et_al_2014, Castellanos_et_al_2014, Yanbay_et_al_2014, Nittrouer_et_al_2014, Freeman_et_al_2017, Boonen_et_al_2021}. Additionally, it implied that given such a large amount of between variability, the statistical models might have a harder time delimiting the \textit{hearing status} groups location on the intelligibility scale.

Finally, for the variability of the entropy replicates, the posterior estimates reveal a reasonable finding: the amount of variability at the replicates level is even larger than the one observed at the children level (bottom panel of figure \ref{fig:variability}). The latter implies there is significant error in measuring speech intelligibility with the entropy replicates. This further emphasize the difficulty of producing unequivocal  inferences, in respect to the intelligibility levels of the \textit{hearing status} groups.
%
%
\begin{figure}[!h]
	\centering
	\begin{tabular}{@{}@{}}
	\includegraphics[trim=0 15 0 50, clip, width=0.65\textwidth]{posterior_predictive_real2_1.pdf} \\
	%trim=left lower right upper
	\includegraphics[trim=0 15 0 40, clip,, width=0.65\textwidth]{posterior_predictive_real2_2.pdf}
	\end{tabular}
	\caption[Model 10, posterior predictive: \textit{true} entropy and \textit{speech intelligibility} scales]{Model 10, posterior predictive: \textit{true} entropy and \textit{speech intelligibility} scales. Colored circles represent mean values per \textit{hearing status} group, lines depict $95\%$ highest posterior density intervals (HDPI). Horizontal discontinuous lines represent the marginal average for the \textit{hearing status} group.}
	\label{fig:predictive2}
\end{figure}
%
%
%###############################
\subsection{The speech intelligibility scale} \label{sS:results_scales}
%###############################
%
As previously stated, one of the main benefits of the current methodology is that we can `construct' the \textit{intelligibility} score of the sampled children, which in turn, allow us to test different hypothesis and even make individual comparisons at the appropriate level. Refer to supplementary section \ref{sSA:SI} for the appropriate interpretation of the score.

Figure \ref{fig:predictive2} depicts the estimated \textit{true} entropy and \textit{speech intelligibility} scores per child. Three points can be highlighted from the figure. First, we can observe that on average NH children have a modest higher level of intelligibility (lower entropy) than the HI/CI counterparts, as we can clearly assess from the horizontal discontinuous lines. However, given the children's \textit{intelligibility} variability, we cannot reject the statistical hypothesis that HI/CI have similar levels of intelligibility than the NH children, as one can notice from the overlapping dispersion of the colored circles on both groups. Lastly, the figure reveals there is some inherited uncertainty on the score estimates, resulting from using the entropy replicates as a tool for measuring \textit{speech intelligibility}. The latter can be observed from the vertical lines describing the highest posterior density intervals. We argue that this last trait of the method is one of its main strengths, as it allow us to not only make individual comparisons, but also consider their inherited (un)certainty.
%
%
%###############################
\subsection{Posterior predictive} \label{sS:results_posterior}
%###############################
%
The posterior predictive results provide us with a tool to assess how good our models were to reconstruct the observed data. Figure \ref{fig:predictive1} shows that our full interaction model managed to recreate the data, while also provided the (un)certainty around the entropy replicates. The latter was shown in two ways: (i) through the point estimates and HPD intervals for the \textit{true} entropies (red points and lines), and (ii) though the expected estimated marginal distribution for the entropy replicates (thick and thin solid distribution lines).
%
\begin{figure}[!h]
	\centering
	\includegraphics[width=0.65\linewidth]{posterior_predictive_real1.pdf}
	\caption[Model 10, posterior predictive: entropy replicates, \textit{true} entropy, and distributions]{Model 10, posterior predictive: entropy replicates, \textit{true} entropy, and distributions. Colored points represent the entropy replicates per \textit{hearing status} group. Red points with lines represent the mean \textit{true} entropy with $95\%$ highest posterior density interval (HPDI). Thick solid line represents the marginal distribution, thin solid lines depicts $100$ random posterior samples for the distributions.}
	\label{fig:predictive1}
\end{figure}
%
%
%###############################
\subsection{Influential observations} \label{sS:results_outliers}
%###############################
%
Finally, implementing the statistical models under the Bayesian framework provide us with one last advantage: we were able to identify influential observations that might sway our estimates, under our model framework \citep{McElreath_2020}. We argue this method is preferable, as the identification of influential observations, through preliminary or univariate procedures, might lead to erroneous exclusion of information, ultimately damaging our research inferences. See supplementary section \ref{ssSA:preprocessing} for more details.

Figure \ref{fig:outliers} provides an overview of the identified high-leverage observations (pair child, utterance). Model ten reveals some observations for children 8, 10, and 15 can be deemed prominent (the first two were NH children, while the last was a HI/CI analogue). 

After a careful inspection, the only similitude these observations shared, was that all of them registered the lowest replicate values observed in the sample ($0.007$ for child 8, and $0$ for the remaining two). Therefore, it makes sense our model found these points `surprising', as they are mostly out of the expected range for the outcome. Notice additionally, that even with their presence, no further evidence was found to favor \textit{robust} over other more parsimonious models (see supplementary section \ref{ssSA:model_selection}).
%
\begin{figure}[!h]
	\centering
	\includegraphics[width=0.6\linewidth]{outliers.pdf}
	\caption[Model 10, influential observations]{Model 10, influential observations. Pairs child, utterance are reported for specific observations. Thin and thick discontinuous lines indicate the minimum and extreme recommended threshold of identification.}
	\label{fig:outliers}
\end{figure}
%
%