%%%%%%%%%%%%%%%%%%%%%%%%%%%%%%%%
\section{Results} \label{S:results}
%%%%%%%%%%%%%%%%%%%%%%%%%%%%%%%%
%
%###############################
\subsection{About the hierarchy of variability} \label{sS:results_variability}
%###############################
%
As expected, three hierarchies of variability were present in our data: the blocks and children's random effects, as well as, the variability of the entropy replicates.

Evidence from the posterior estimates reveal the block random effects explained a small amount of variability in the data (top panel of figure \ref{fig:variability}), and its inclusion/exclusion in the model did not change the parameter estimates. The previous implies the experiment was correctly `set up', as the series in which the utterances were transcribed did not explain a significant amount of variation, nor its exclusion biased the parameter estimates.

On the contrary, we observe a significantly larger variability between children's \textit{speech intelligibility}, more precisely, more than three times the block effects variability (middle panel of figure \ref{fig:variability}). The previous corroborated preliminary evidence on the matter \cite{Young_et_al_2002, Peng_et_al_2004, Montag_et_al_2014, Castellanos_et_al_2014, Yanbay_et_al_2014, Nittrouer_et_al_2014, Freeman_et_al_2017, Boonen_et_al_2021}. Additionally, it implied that given such a large amount of between variability, the statistical models might have a harder time comparing the two \textit{hearing status} groups, in respect to their location on the intelligibility scale.

Finally, for the variability of the entropy replicates, the posterior estimates reveal a reasonable finding: the amount of variability at the replicates level is even larger than the one observed at the children level (bottom panel of figure \ref{fig:variability}). The latter implies there is significant error in measuring speech intelligibility with the entropy replicates. This further emphasize the difficulty of producing unequivocal  inferences, in respect to the intelligibility levels of the \textit{hearing status} groups.
%
\begin{figure}[!h]
	\centering
	\includegraphics[width=0.5\linewidth]{variability_plot.pdf}
	\caption[Posterior predictive: hierarchy of variability in the data]{Posterior predictive: hierarchy of variability in the data. Distributions are plotted at the entropy replicates scale, considering three different average entropy values: $\mu=0.2$, $\mu=0.5$, and $\mu=0.8$ (discontinuous lines). Thick solid lines represent the marginal distribution, thin solid lines depict $100$ random posterior distribution samples.}
	\label{fig:variability}
\end{figure}
%
%###############################
\subsection{About our hypothesis} \label{sS:results_hypothesis}
%###############################
%
The current research used the Information-Theoretic Approach \citep{Anderson_2008, Chamberlain_1965} for model selection and inference. The application of the approach required: (i) the expression of the research hypothesis into statistical models, (ii) the selection of the most plausible models, and (iii) to produce inferences based on one or multiple selected models. The first requirement of the approach is covered in sections \ref{sS:causal_frame} and \ref{sS:stat_analysis}, and expanded in supplementary section \ref{sSA:model_details}. Here we use the results of the second requirement, detailed in the supplementary section \ref{ssSA:model_selection}, to produce our inferences. 

As detailed in the aforementioned supplementary section, the final conclusions of our research will be drawn from the comparisons of two models: (i) a no interaction model, with one estimated `sample size' (model 3), and (ii) a full interaction model, with one estimated `sample size' (model 10). The former is selected because it is the model with highest probabilistic support. The latter is considered because it encompasses the remaining highest supported models. Furthermore, no \textit{robust} model is inspected, as we prefer a more parsimonious depiction of our hypothesis.
%
\begin{table}[h!]
	\centering
	\begin{tabular}{|cccccccccccc|} 
		\hline
		& \multicolumn{3}{c}{} & \multicolumn{2}{c}{CI} & & \multicolumn{2}{c}{HPDI} & & \multicolumn{2}{c|}{}\\[0.5ex]
		\cline{5-6} \cline{8-9}
		Parameter & Mean & SD & & $2.5\%$ & $97.5\%$ & & $2.5\%$ & $97.5\%$ & & n eff. & Rhat \\[0.5ex] 
		\hline\hline
		\multicolumn{12}{|l|}{ \textbf{Model 3: No interaction (one `size')} } \\
		\multicolumn{1}{|r}{ \texttt{a} } & & & & & & & & & & & \\
		\multicolumn{1}{|r}{ \texttt{bP[HI/CI]} } & & & & & & & & & & & \\
		\multicolumn{1}{|r}{ \texttt{bA} } & & & & & & & & & & & \\
		\hline
		\multicolumn{12}{|l|}{ \textbf{Model 10: Full interaction (one `size')} } \\
		\multicolumn{1}{|r}{ \texttt{a} } & & & & & & & & & & & \\
		\multicolumn{1}{|r}{ \texttt{bP[HI/CI]} } & & & & & & & & & & & \\
		\multicolumn{1}{|r}{ \texttt{bAHS[NH]} } & & & & & & & & & & & \\
		\multicolumn{1}{|r}{ \texttt{bAHS[HI/CI]} } & & & & & & & & & & & \\
		\hline
		\multicolumn{12}{l}{\footnotesize{CI = compatibility interval}} \\
		\multicolumn{12}{l}{\footnotesize{HDPI = highest posterior density interval}} \\
		\multicolumn{12}{l}{\footnotesize{n eff. = effective number samples}} \\
		\multicolumn{12}{l}{\footnotesize{Rhat = Gelman-Rubin diagnostic}} \\
	\end{tabular}
	\caption[Characteristics of selected children]{Characteristics of selected children.}
	\label{tab:characteristics}
\end{table}

\textcolor{red}{work in progress}

\begin{comment}
“simplest” model (E_NC2b) provides
(preliminar) evidence on,
the higher the unaided PTA the
lower the child’s SI (bP[2])
(based on power analysis, we can be
sure is a small effect)
no apparent statistical difference
between NH and HI=CI children
(but this requires a CONTRAST)
for each “hearing” year, the SI
increases in approx 0:40 logits
(effect larger than the assumed in
power analysis)


however, the “interaction” model
(E_NC5b3) shows similar results on,
the small (still non-significant)
effects of the unaided PTA on the
child’s SI
similar explained variability across
levels and blocks
(similar to the “simplest” model)
but “mild” evidence of prevalent
interactions,
SI means for HI=CI per E,
aEHS[2; 2] (Genetic) vs
aEHS[3; 2] (CMV)
different SI evolution for NH vs
HI=CI children, per unit A
(bAHS),

within the “interaction” model,
the size of the data within groups
from combinations of E and HS,
does not allow to reject the
contrasts’ null hypothesis,
similar result is observed on the
bAHS contrast
(because the effect is small, compared
to children’s variability)
but we still observe differences
between NH and HI/CI, and even
within HI/CI by E,
therefore we decide to keep the
(E_NC5b3) model
\end{comment}



\begin{comment}
	Following the successful and comprehensive analysis in \citet{vanDaal_2020} and \citet{Lesterhuis_2018}, 
	
	Notice the model depicted in panel (a) is interested on (what we can call) \textit{total effects}, i.e. the effects of the hearing characteristics, not independent from the effects of the hearing apparatus (cochlear implant or hearing aid). This is important to understand for two reasons. Since a hearing apparatus is fitted onto a child depending on aspects such as the locus and severity of his(her) hearing impairment \citep{Korver_et_al_2017}: (1) such specific children's characteristics could confound the (beneficial) effects of using specific hearing apparatuses, while (2) because children are selected from a convenient sample, not representative of their respective populations (see section \ref{s_sect:children}), the need to control for such characteristics is paramount, if we seek to obtain effects that can generalize better and beyond our sample\footnote{follow the \textit{notes} folder, to see a graphical though experiment.}.
	
	Considering the previous, we propose the model depicted in panel (b), where we control for the possible confounding variables etiology ($E_{i}$), \textcolor{blue}{as a proxy of locus}, and unaided PTA ($PTA_{i}$), as a proxy for hearing impairment severity. In that sense, the model would estimate (what we can call) the \textit{direct effects} of the hearing apparatus, independent of the children's characteristics.
\end{comment}
%
%
%###############################
\subsection{The speech intelligibility scale} \label{sS:results_scales}
%###############################
%
\textcolor{red}{work in progress}
%
\begin{figure}[!h]
	\centering
	\includegraphics[width=0.7\linewidth]{posterior_predictive_real2.pdf}
	\caption[Posterior predictive: \textit{true} entropy and \textit{speech intelligibility} scales]{Posterior predictive: \textit{true} entropy and \textit{speech intelligibility} scales. Circles represent mean values, lines depict $95\%$ highest posterior density intervals (HDPI). Horizontal discontinuous lines represent the marginal average for the \textit{hearing status} groups.}
	\label{fig:predictive2}
\end{figure}
%
%
%###############################
\subsection{Posterior predictive} \label{sS:results_posterior}
%###############################
%
\textcolor{red}{work in progress}
%
\begin{figure}[!h]
	\centering
	\includegraphics[width=0.7\linewidth]{posterior_predictive_real1.pdf}
	\caption[Posterior predictive: entropy replicates]{Posterior predictive: entropy replicates. Red point with lines represent the mean \textit{true} entropy with $95\%$ highest posterior density interval (HPDI). Thick solid line represents the marginal distribution, thin solid lines depicts $100$ random posterior distribution samples.}
	\label{fig:predictive1}
\end{figure}
%
%
%###############################
\subsection{Influential observations} \label{sS:results_outliers}
%###############################
%
\textcolor{red}{work in progress}
%
\begin{figure}[!h]
	\centering
	\includegraphics[width=0.5\linewidth]{outliers.pdf}
	\caption[Influential observations]{Influential observations. Pairs (child, utterance) are reported for specific observations.}
	\label{fig:outliers}
\end{figure}
%
%