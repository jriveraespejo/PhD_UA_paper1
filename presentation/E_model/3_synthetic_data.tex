%%%%%%%%%%%%%%%%%%%%%%%%%%%%%%%%%%%%%%%%%%%%%%%%%%%%%%%%%%%
\section{Synthetic data generation}
%%%%%%%%%%%%%%%%%%%%%%%%%%%%%%%%%%%%%%%%%%%%%%%%%%%%%%%%%%%
%
%
\begin{frame}[t, negative]
	\subsectionpage
\end{frame}
%
%
\begin{frame}
	{Intervention}
	%
	\begin{columns}
		%
		\begin{column}{0.5\textwidth}
			%
			\begin{itemize}
				%
				\item Purpose: to keep a control on all the factors responsible for the outcome's variation \texcolort{blue}{(understand the system)}.
				%
				\item It is modeled by modifying the structural model (and causal diagram).
				%
				\item remember: $\pmb{V}=\{Z,X,Y\}$, $\pmb{U}=\{U_{z},U_{x},U_{y}\}$, and $\pmb{F}=\{f_{z},f_{x},f_{y}\}$.
				%
				\item Intervention on $X$ can be written in do-calculus\footnote{we are not delving into this (the usual suspects \cite{Pearl_1988, Pearl_2009, Pearl_et_al_2016, Pearl_et_al_2018})} as: $P(\pmb{V} \; | \; do(X=x))$.
				%
			\end{itemize}
			%
		\end{column}
		%
		\begin{column}{0.5\textwidth}  
			%
			\begin{equ}
				%
				M = \left\{ \begin{aligned} 
					Z \leftarrow & \; f_{z}(U_{z}) \\
					X \leftarrow & \; f_{x}(U_{x}) \\
					Y \leftarrow & \; f_{y}(X, Z, U_{y}) \\
					U \sim & \; P(\pmb{U})
				\end{aligned} \right
				%
				\caption*{(a) structural model}
			\end{equ}
			%
			\begin{figure}
				%
				\begin{tikzpicture}
					% nodes
					\node[formula] at (-2,0) {$x$};
					\node[formula] at (-1,-0.3) {$X$};
					\node[formula] at (1,1.5) {$U_{z}$};
					\node[formula] at (0,1) {$Z$};
					\node[formula] at (2,0) {$U_{y}$};
					\node[formula] at (1,-0.3) {$Y$};
					
					% paths
					\draw [{Circle [open]}-{latex}](-1.7,0)--(-1,0); % Ux->X
					\draw [{Circle}-{latex}](-1,0)--(0.9,0); % X->Y
					\draw [{Circle [open]}-{latex}{Circle}](1.7,0)--(0.9,0); % Uy->Y
					\draw [{Circle}-{latex}](0,0.8)--(0.9,0.1); % Z->Y
					\draw [{Circle [open]}-{latex}](0.9,1.3)--(0.1,0.8); % Uz->Z
				\end{tikzpicture}
				\caption*{(b) causal diagram}
			\end{figure}
			%
		\end{column}
		%
	\end{columns}
	%
\end{frame}
%
%
